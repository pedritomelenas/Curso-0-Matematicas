\documentclass[14pt]{amsart}
\usepackage[utf8]{inputenc}
\usepackage{btxdockit}
\usepackage[margin=1in,screen]{geometry}
\usepackage[dvipsnames]{xcolor}
\usepackage{setspace}
\usepackage{graphicx}
\usepackage{polynom}
\usepackage{tikz}
\usepackage{xspace,colortbl}
%\usepackage[screen,gray]{pdfscreen} \margins{.75in}{.75in}{.75in}{.75in} \screensize{6.25in}{8in}

\onehalfspacing

\usepackage{amsmath,amssymb,amsthm}
\allowdisplaybreaks[1]

%\usepackage[latin1]{inputenc}
\usepackage[spanish]{babel}

\usepackage{xspace,colortbl}





\usepackage[bitstream-charter]{mathdesign}
\usepackage[T1]{fontenc}

\usepackage{comment}
\usepackage[active,tightpage]{preview}


\setlength\PreviewBorder{5pt}%

\begin{document}

\thispagestyle{empty}

\begin{comment}

\begin{preview}
$
\displaystyle{\polyhornerscheme[x=\frac{2}{3},tutor=true,arraycolsep=0.6cm,stage=1]{2x^4 - \frac{1}{3} x^3 + \frac{3}{2}x^2 - 1}}
$
\end{preview}


El coeficiente líder del cociente coincide con el coeficiente líder del dividendo:



\begin{preview}
$
\displaystyle \polyhornerscheme[x=\frac{2}{3},tutor=true,arraycolsep=0.6cm,stage=2]{2x^4 - \frac{1}{3} x^3 + \frac{3}{2}x^2 - 1}
$
\end{preview}


Multiplicamos $2$ por $\frac{2}{3}$

\begin{preview}
\(
\displaystyle \polyhornerscheme[x=\frac{2}{3},tutor=true,arraycolsep=0.6cm,stage=3]{2x^4 - \frac{1}{3} x^3 + \frac{3}{2}x^2 - 1}
\)
\end{preview}


y lo sumamos a $-\frac{1}{3}$
\

\begin{preview}
\(\displaystyle\polyhornerscheme[x=\frac{2}{3},tutor=true,arraycolsep=0.6cm,stage=4]{2x^4 - \frac{1}{3} x^3 + \frac{3}{2}x^2 - 1}
\)
\end{preview}


Repetimos el proceso hasta calcular todos los coeficientes del cociente,

\begin{preview}
\(
\polyhornerscheme[x=\frac{2}{3},tutor=true,arraycolsep=0.6cm,stage=8,tutorlimit=4]{2x^4 - \frac{1}{3} x^3 + \frac{3}{2}x^2 - 1}\)
\end{preview}

inalmente calculamos el resto con el mismo procedimiento,

\begin{preview}
\(\polyhornerscheme[x=\frac{2}{3},tutor=true,arraycolsep=0.6cm,stage=10,tutorlimit=2,resultleftrule=true,resultbottomrule=true]{2x^4 - \frac{1}{3} x^3 + \frac{3}{2}x^2 - 1}\)
\end{preview}

FPor tanto el cociente es $2x^3 + x^2 + \frac{13}{6}x + \frac{13}{9}$ y el resto $-\frac{1}{27}$.
\end{ejemplo}

\begin{ejemplo}
Algunos ejemplos más. $-\frac{1}{2}x^3 + \frac{5}{3} x^2 + \frac{3}{7}x^2 + \frac{3}{4}$ entre $x+2$:

\begin{preview}
\(\polyhornerscheme[x=-2,tutor=true,stage=8,arraycolsep=0.6cm
,tutorlimit=8,resultleftrule=false,resultbottomrule=true]{-\frac{1}{2}x^3 + \frac{5}{3} x^2 + \frac{3}{7}x^2 + \frac{3}{4}}\)
\end{preview}


$x^5 - 3x^4 + 2 x^2 - 3 x - 9$ entre $x-3$:


\begin{preview}
\(
\polyhornerscheme[x=3,resultleftrule=true,resultbottomrule=true]{x^5 - 3x^4 + 2 x^2 - 3 x - 9}
\)
\end{preview}


$x^4 - 4 x^2  +4$ entre $x+1$:



\begin{preview}
\(
\polyhornerscheme[x=-1,resultleftrule=true,resultbottomrule=true]{x^4 - 4 x^2  +4}
\)
\end{preview}


\end{ejemplo}

\begin{ejercicio}
Calcula cociente y resto obtenidos al dividir
\begin{itemize}
\item $2x^3 + 5 x^2 - 3 x + 2$ entre $x-2$,
\item $x^5 + \frac{1}{2} x^3 - \frac{2}{5} x + \frac{4}{3}$ entre $x + \frac{2}{5}$,
\item $\frac{4}{5} x^3 - \frac{4}{7} x^2 - 3 x - 2$ entre $x-1$.
\end{itemize}
\end{ejercicio}


\section{Raices de un polinomio y factorización}

Dado un polinomio $p(x)=a_0+a_1x+a_2x^2+ \dots +a_nx^n$, una {\it raiz} es un número $\alpha$ tal que al evaluar el polinomio en $\alpha$ (sustituir la indeterminada por dicho valor) el resultado es $0$, es decir, $p(\alpha)=0$. 


El siguiente resultado es una aplicación del algoritmo de la división de polinomios, en efecto, la división de $p(x)$ entre $x-\alpha$ nos proporciona la fórmula
$$p(x)=q(x)\cdot (x-\alpha) + r$$
donde $r$ es un polinomio de grado menor que 1, es decir, es un número. 


\begin{proposicion}
Para un polinomio $p(x)$ son equivalentes las siguientes afirmaciones:
\begin{enumerate}
\item $\alpha$ es una raiz de $p$,
\item $\alpha$ es una solución de la ecuación $p(x)=0$,
\item $p(x)=q(x)(x-\alpha)$.
\end{enumerate}
\end{proposicion}

A los divisores de la forma $x-\alpha$ de un polinomio se les denomina {\it factores lineales}. Es de gran utilidad obtener la descomposición en factores lineales de un polinomio, o lo que es lo mismo obtener todas las soluciones de una ecuación polinómica. La resolución de algunos tipos de estas ecuaciones es tratado en el apartado ''Ecuaciones polinómicas'' de la siguiente sesión, sin embargo, es interesante el siguiente método para calcular las posibles raices enteras de un polinomio con coeficientes también enteros.

Supongamos que tenemos un polinomio de grado $n$ con coeficientes enteros, digamos 

$$a_nx^n +\dots +a_1x+a_0$$

si tuviese una raiz entera, $\alpha$, entonces el polinomio sería exactamente el producto de otro polinomio de grado $n-1$, también con coeficientes enteros, y el factor lineal correspondiente, es decir:

$$a_nx^n +\dots +a_1x+a_0=(x-\alpha)\cdot(a_nx^{n-1}+\dots +b_1x+b_0)$$

El hecho de que los coeficientes del cociente son enteros puede deducirse comprobando que en el método de Ruffini todas las operaciones son sumas y productos de números enteros. Ahora observamos que los términos de grado 0 en ambos miembros de la igualdad deben ser iguales 

$$a_0=\alpha \cdot b_0$$

y así la raiz $\alpha$ es un divisor del término independiente, y como un número entero distinto de cero tiene una cantidad finita de divisores, puede explorarse el conjunto completo de divisores en busca de las posibles raices.

\begin{ejemplo}
\item Para encontrar todas las raices enteras del polinomio
$$ x^4-7x^2+7x-2$$
observamos que como el término independiente es $-2$, sus posibles divisores son $1,-1,2,-2$ así que probamos si alguno de ellos es raiz. Esto puede hacerse realizando la división por Ruffini o simplemente evaluando el polinomio
$$\begin{array}{l}
p(1)=1^4-7\cdot 1^2+7\cdot 1-2=-1\\
p(-1)=(-1)^4-7\cdot (-1)^2+7\cdot (-1) -2=1+7-7-2=-1\\
p(2)=2^4-7\cdot 2^2+7\cdot 2-2= 16-28+14-2=0\\
p(-2)=(-2)^4-7\cdot (-2)^2+7\cdot (-2)-2= 16-28-14-2=-28\\
\end{array}
$$
Así que $2$ es raiz del polinomio, para factorizar dividimos por $x-2$:
\end{comment}
 
\begin{preview}
\(
\polyhornerscheme[x=2,resultleftrule=true,resultbottomrule=true]{x^4 - 7x^2 + 7 x - 2}
\)
\end{preview}

\begin{comment}

El cociente es $x^3+2x^2-3x+1$, que solo podría tener como raices enteras a $1$ y $-1$, y ya hemos comprobado que no lo son del polinomio dado, y por tanto tampoco del cociente.

\end{ejemplo} 

\begin{observacion}
Dado un polinomio con coeficientes racionales sus raices son las mismas que las de un polinomio con coeficientes enteros que se obtiene al multiplicar el original por el $mcm$ de todos los denominadores que aparezcan en los coeficientes. Por ejemplo, el polinomio $ x^3+\frac{1}{6}x^2+ \frac{1}{5}x-1$ tiene las mismas raices que $30x^3+5x^2+6x-30$, así que si tuviese raices enteras estarían entre los divisores de $30$.
\end{observacion}

\begin{ejercicio} Factoriza, cuando sea posible, los siguientes polinomios:
\begin{enumerate}
\item $x^3+6x+9x$
\item $x^4-2x^2+1$
\item $x^3+3x^2-4x-12$
\item $x^5+20x^3 +100x$
\item $2x^5-32x$
\item $\frac{2}{5}x^5- \frac{6}{5}x^4+\frac{14}{15}x^2$
\end{enumerate}
\end{ejercicio}
\end{comment}

\end{document}
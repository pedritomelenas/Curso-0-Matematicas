\documentclass[13pt]{scrartcl}
\usepackage{btxdockit}
\usepackage[margin=1in,screen]{geometry}
\usepackage[dvipsnames]{xcolor}
\usepackage{setspace}
\onehalfspacing

\KOMAoptions{numbers=noenddot}
\addtokomafont{paragraph}{\spotcolor}
\addtokomafont{section}{\spotcolor}
\addtokomafont{subsection}{\spotcolor}
\addtokomafont{subsubsection}{\spotcolor}
\addtokomafont{descriptionlabel}{\spotcolor}

\usepackage{fancyhdr}

\pagestyle{fancy}

\renewcommand{\sectionmark}[1]{%
 \markboth{#1}{}}

\lfoot{\thepage}
\cfoot{}
\rfoot{\footnotesize  Jerónimo Alaminos Prats, José Extremera Lizana, Pilar Muñoz Rivas\copyright}

\usepackage{amsmath,amssymb,amsthm}
\allowdisplaybreaks[1]


\usepackage[utf8]{inputenc}
\usepackage[spanish]{babel}

\usepackage{xspace,colortbl}



\usepackage{multicol}


\swapnumbers
\newtheorem{teorema}{Teorema}[section]
\newtheorem{lema}[teorema]{Lema}
\newtheorem{proposicion}[teorema]{Proposici\'on}
\newtheorem{corolario}[teorema]{Corolario}

\theoremstyle{definition}
\swapnumbers
\newtheorem{nota}{Nota}
\newtheorem*{nota*}{Nota}        
\newtheorem{definicion}[teorema]{Definici\'on}         
\newtheorem{observacion}{Observaci\'on}
\newtheorem*{observacion*}{Observaci\'on}
\newtheorem{ejemplos}{Ejemplos}
\newtheorem*{ejemplos*}{Ejemplos}
\newtheorem{ejemplo}{Ejemplo}
\newtheorem*{ejemplo*}{Ejemplo}
\newtheorem{ejercicio}{Ejercicio}


\usepackage{cancel}



\usepackage{setspace}
\onehalfspacing

\usepackage[bitstream-charter]{mathdesign}
\usepackage[T1]{fontenc}


\begin{document}

\title{Inecuaciones}

{\huge \textsf{\textbf{\color{Peach} Inecuaciones}}}

\vspace{1em}

Las inecuaciones son  expresiones algebraicas relacionadas no por la igualdad ($=$) como en 
las ecuaciones, sino por símbolos de desigualdad, como  ``$\le$'', ``$\ge$'', ``$<$'' o ``$>$''. 

Las inecuaciones con solución tienen, en general, infinitas soluciones. El proceso para 
resolverlas es similar al que se sigue para resolver una ecuación, siendo cuidadoso con las 
reglas de cálculo cuando aparecen desigualdades. Por ejemplo, por destacar una de estas 
reglas, si en algún momento de la resolución es preciso multiplicar o dividir ambos 
miembros de la inecuación por un número negativo, recordamos que la desigualdad 
se invierte.



\section{Inecuaciones lineales con una incógnita}





Son las inecuaciones más sencillas que se nos pueden presentar.

\begin{ejemplo} Vamos a resolver la inecuación $2x+5 \ge x-5$. 
\begin{align*}
2x+5 & \ge x-5 \\
\intertext{pasamos la $x$ del miembro izquierdo a la derecha, y los coeficientes independientes los dejamos a la izquierda}
2x-x & \ge -5-5 \\
x &\ge -10
\end{align*}
\end{ejemplo}
Por tanto la solución de la inecuación son todos los números reales que verifican ser mayores o iguales a -10, es decir, las soluciones se encuentran en el intervalo $[-10,+\infty[$.

\begin{ejercicio} $ $
\begin{enumerate}
\item[a)] Resuelve la inecuación: $4x-5 < 7x+8$
\item[b)] Resuelve la inecuación: $2(-x+5)-(6+x)>4+2(x-3)$  y razona si los números $-3, -1, 0$ y $ 1$ son solución de la misma.
\item[c)] Ídem al ejercicio anterior con la inecuación:$\dfrac{3x}{5}-\dfrac{11x}{10} \ge 1-\dfrac{6-x}{4}$
\item[d)] Resuelve: $(x+1)^2-(x-1)^2+12 \ge 0$
\end{enumerate}
\end{ejercicio}

\section{Sistemas lineales de dos inecuaciones con una incógnita}

Vamos a plantear ahora sistemas de dos inecuaciones lineales con una incógnita. Veremos que 
las soluciones, si las hay, vuelven a ser infinitas.

\begin{ejemplo} Consideremos el sistema 
\[
\left\{ 
\begin{aligned} x-3 &>0 \\
              2x & < 8 
\end{aligned}
\right.
\]

Para resolverlo, resolvemos cada inecuación de forma independiente.
\[
\Biggr\{
\begin{aligned}
 x & >3 \\
 x & <4 
\end{aligned}
\]
con lo que la solución es  $3 < x <4$. Es decir, las soluciones de este sistema son todos los números reales que están en el intervalo $]3,4[$.

Hay que tener en cuenta que un sistema de inecuaciones de este tipo puede tener solución (sistema compatible) o puede no tenerlo (sistema incompatible).
\end{ejemplo}

\begin{ejercicio} Resuelve los siguientes sistemas de inecuaciones:
\begin{multicols}{3}
\item[a)]$\Biggr\{ \begin{aligned}  3x-6 & \ge 9 \\
                                        4-x & < -5 \end{aligned} $
\item[b)] $\Biggr\{ \begin{aligned} 3x-6 & \le 9 \\
                                      4-x & < -5 \end{aligned} $
\item[c)] $\Biggr\{ \begin{aligned} 8x+7 & < 16-x \\
                                      -3x+5 & >2x \end{aligned} $
\end{multicols}
\end{ejercicio}



\section{Inecuaciones de segundo grado}

También podemos plantear y resolver inecuaciones a partir de un polinomio de grado 2; esto 
es, una inecuación del tipo $P(x) \le Q(x)$, o, $P(x) <Q(x)$, donde ambas expresiones, $P$ 
y $Q$, son polinomios de grado 2. Para resolverlas, simplificamos coeficientes hasta conseguir 
una inecuación de 2º grado comparada con el cero. Después factorizamos el polinomio 
resultante (utilizando la resolución de una ecuación de 2º grado) y discutimos casos. Vemos 
todo este proceso con un ejemplo:

\pagebreak[4]

\begin{ejemplo} Consideramos la inecuación: $x^2-3x-3<-4x-1$. Pasamos todos los 
coeficientes a un miembro, por ejemplo al izquierdo y nos queda:
\begin{align*}
x^2-3x+4x-3+1 & <0 \\
x^2+x-2&<0 \\
\intertext{factorizamos el polinomio resolviendo la ecuación $x^2+x-2=0$}
(x+2)(x-1) &<0
\end{align*}
Llegados a este punto, discutimos casos para encontrar la solución de la inecuación.
\begin{align*}
\text{Si} \ x < -2 & \ \implies \ (x+2)(x-1)>0 \\
\text{Si} \ -2 < x <1 & \ \implies \ (x+2)(x-1)<0 \\
\text{Si} \ x > 1 & \ \implies \ (x+2)(x-1)>0 
\end{align*}
Por tanto, la solución de la inecuación la forman los números $x$, que verifican $-2<x<1$, es decir, el intervalo $]-2,1[$.
\end{ejemplo}

\pagebreak[4]

\begin{ejercicio} Resolver cada una de las inecuaciones siguientes:
\begin{multicols}{3}
\item[a)] $x^2+2x+4\ge 2x+5$
\item[b)] $16x-x^2\ge16 $
\item[c)] $2x^2-2x+9\le x^2-5x+7$
\end{multicols}
\end{ejercicio}



\section{Inecuaciones del tipo $\frac{P(x)}{Q(x)} \le R(x)$}
Cuando se resuelven este tipo de inecuaciones no hay que caer en el error de ``presuponer" 
nada. Intentemos explicarnos mejor con el siguiente ejemplo.
\begin{ejemplo} Consideramos la inecuación: $\frac{x^2+1}{x-1} \le x$. Es claro que el 
punto $x=1$ no es solución, ya que anula al denominador. Entonces, en primer lugar, 
quitamos el denominador multiplicando ambos miembros por el denominador $x-1$:
\begin{align*}
x^2+1 & \le x^2-x \\
1 & \le -x \\
\intertext{Multiplicando ambos miembros de la inecuación por $-1$ y la desigualdad  se invierte}
-1 & \ge x
\end{align*}
Parecería entonces que la solución es el conjunto $]-\infty,-1] $. Sin embargo, es fácil comprobar que 
el punto $x=0$  es solución de inecuación y,  evidentemente, no pertenece al conjunto 
solución que acabamos de dar. ?`Qué ha pasado entonces? Ha pasado que, en el primer 
paso que hemos dado (multiplicar ambos miembros de la inecuación por el denominador 
$x-1$), hemos ``presupuesto" que $x-1>0$, es decir, que $x>1$. Por lo que el 
conjunto solución que hemos obtenido por ahora no tiene sentido con la suposición 
que hemos hecho de $x>1$. Por tanto, como en el caso $x-1>0$ no habría solución,  
tenemos que contemplar la posibilidad de que $x-1<0$, es decir, de que $x$ fuera menor 
que $1$. Consideramos este caso ahora, con lo que al multiplicar por el denominador, la 
desigualdad se invierte.
\begin{align*}
x^2+1 & \ge x^2-x \\
1 & \ge -x \\
x & \ge -1
\end{align*}
Por tanto, la solución está formada por los puntos $x$ tales que  $x<1$ y además $x \ge -1$. Es decir, el intervalo $[-1,1[$.
\end{ejemplo}

\pagebreak[4]

\begin{ejercicio} Resolver cada una de las inecuaciones siguientes:
\begin{multicols}{3}
\item[a)] $\frac{2x-3}{x-4} \ge 3$
\item[b)] $\frac{x-1}{x+1} <1$
\item[c)] $\frac{2x^2-4}{2x+1} \ge x-1$
\end{multicols}
\end{ejercicio}




\end{document}\grid
